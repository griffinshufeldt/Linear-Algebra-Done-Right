\documentclass[12pt]{article} % Sets the font size to 12pt
\usepackage{amsmath,amsthm,amssymb,color,latexsym}
\usepackage[margin=1in]{geometry}
\usepackage[all]{xy}
\usepackage{multicol}
\usepackage{booktabs}
\usepackage{mathtools}
\usepackage{graphicx}
\usepackage{natbib}

\newtheorem{problem}{}[subsection]

\newenvironment{solution}[1][\it{Proof}]{\textbf{#1. } }{$\square$}

\begin{document}                  % DO NOT DELETE THIS LINE

\title{Linear Algebra Done Right, 4th Edition Solutions}

\author{Griffin Shufeldt}
\date{ }

\maketitle                        % DO NOT DELETE THIS LINE
\newpage

\section*{Chapter 1: Vector Spaces}
\subsection*{Exercises 1A: $R^n$ and $C^n$}
\textbf{1.} Show that $\alpha + \beta = \beta + \alpha$ for all $\alpha,\beta \in C$ \\

\begin{solution}
\begin{align*}
    & \alpha, \beta \in C \rightarrow \alpha = a + bi \text{ and } \beta = c + di \text{ where } a,b,c,d \in R & \\
    & \text{LHS: } & \\
    & \alpha + \beta = (a + bi) + (c + di) = (a+c) + (b+d)i & \\
    & \text{RHS: } & \\
    & \beta + \alpha = (c+di) + (a+bi) = (c+a) + (d+b)i & \\
    & \text{Because $a,b,c,d \in R$, they obey commutativity} & \\
    & \rightarrow \beta + \alpha = (c+a) + (d+b)i = (a+c) + (b+d)i = \alpha + \beta
\end{align*}
\end{solution}

\textbf{2.} Show that $(\alpha + \beta) + \lambda = \alpha + (\beta + \lambda)$ for all $\alpha,\beta,\lambda \in C$ \\

\begin{solution}
\begin{align*}
    &\text{LHS: } & \\
    &\alpha = a+bi, \beta = c+di, \lamba = e+fi &\\
    &(\alpha + \beta) + \lambda = ((a+bi)+(c+di))+(e+fi) = & \\
    &(a+c)+(b+d)i + (e+fi) & \\
    &(a+c+e)+(b+d+f)i  & \\
    & \text{RHS: } & \\
    &\alpha + (\beta + \lambda) = (a + bi) + ((c+di)+(e+fi)) = & \\
    &(a+bi) + ((c+e) + (d+f)i) = (c+e+a) + (d+f+b) & \\
    &\text{By commutativity of real numbers } (\alpha + \beta) + \lambda = \alpha + (\beta + \lambda) 
\end{align*}
\end{solution}

\textbf{3.} Show that $(\alpha\beta)\lambda = \alpha(\beta\lambda)$ for all $\alpha,\beta,\lambda \in C$ \\

\begin{solution}
\begin{align*}
    &\text{LHS: } & \\ 
    &\alpha\beta = (a+bi)(c+di) = (ac - bd) + (ad + bc)i & \\
    &((ac - bd) + (ad + bc)i)\lambda = (ac - bd) + (ad + bc)i)(e+fi) & \\
    &(ace - bde) + (ade+bce)i +(acf - bdf)i + (adf+bcf)(-1) & \\
    &(ace-bde-adf-bcf)+(ade+bce+acf-bdf)i & \\
    &\text{RHS: } & \\ 
    & \alpha(\beta\lambda) = (a+bi)((c+di)(e+fi)) & \\
    & = (a+bi)((ce-df)+(cf+de)i)  & \\
    & = (cea - dfa -cfb -deb) + (cfa+dea +ceb -dfb)i & \\
    & \text{By commutativity: } LHS = RHS &
\end{align*}
\end{solution}

\textbf{4.} Show that $\lambda(\alpha + \beta) = \lambda\alpha + \lambda\beta$ for all $\lambda, \alpha, \beta \in C$ \\

\begin{solution}
\begin{align*} 
    &\text{LHS: } &\\
    &\lambda(\alpha+\beta) = (e+fi)((a+bi) + (c+di)) &\\
    &= (e+fi)((a+c)+(b+d)i) &\\
    &(e+fi)(a+c)+(e+fi)(bi+di) & \\
    &(ea +ce) + (fa +fc)i + (eb + ed)i +(-fb -fd)& \\
    &(ea + ce - fb - fd) + (fa+fc+eb+ed)i &\\
    &\text{RHS: } &\\
    &\lambda\alpha + \lambda\beta = (e+fi)(a+bi) + (e+fi)(c+di) &\\
    &(ea-fb) + (fa+eb)i + (ec-fd)+(fc+ed)i& \\
    &= (ea-fb+ec-fd) + (fa+eb+fc+ed)i &  \\
    &\text{By Commutativity: } LHS = RHS &
\end{align*}
\end{solution}

\textbf{5.} Show that for every $\alpha \in C$ there exists a unique $\beta \in C$ such that $\alpha + \beta = 0$ \\ 

\begin{solution}
\begin{align*}
    &\text{Proving existence: } & \\
    &\alpha = (a+bi), \beta = (-a-bi) \rightarrow \alpha + \beta = 0 & \\
    &\text{Proving uniqueness: assume that $\lambda \ne \beta$ is also an additive inverse} & \\
    &\alpha + \beta = 0 = \alpha + \lambda \rightarrow \lambda = \beta & \\
    &\text{Contradiction, only assumption made was $\lambda \ne \beta$, so $\lambda = \beta$ }
\end{align*}
\end{solution}

\textbf{6.} Show that for every $\alpha \in C$ with $\alpha \ne 0$ there exists a unique $\beta \in C$ such that $\alpha\beta = 1$ \\ 

\begin{solution}
\begin{align*}
    &\text{Proving existence: } & \\
    &\alpha = (a+bi), \beta = \frac{1}{(a+bi} \rightarrow \alpha\beta = 1 & \\
    &\text{Proving uniqueness: assume that $\lambda \ne \beta$ also satisies this property} & \\
    &\alpha\beta = 1 = 1(\alpha\lambda) \rightarrow \beta = \lambda & \\
    &\text{Contradiction, only assumption made was $\lambda \ne \beta$, so $\lambda = \beta$ }
\end{align*}
\end{solution}

\textbf{7.} Show that $\frac{-1 + \sqrt{3}i}{2}$ is a cube root of 1 (TODO) \\ 

\textbf{8.} Find two distinct square roots of $i$ (TODO) \\

\textbf{9.} Find $X \in R^4$ such that (4,-3,1,7) +2x = (5,9-6,8) \\ 

\begin{solution}
\begin{align*}
    & y = 2x \rightarrow (4,-3,1,7) + y = (4+y_1,-3+y_2,1+y_3,7+y_4) & \\
    & y_1 = 1, y_2 = 12, y_3 = -7, y_4 = 1 \rightarrow (4,-3,1,7) + y = (5,9,-6,8) & \\
    & x = \frac{y}{2} = (1,12,-7,1)/2 = (\frac{1}{2},6,\frac{-7}{2},\frac{1}{2}) & 
\end{align*}
\end{solution}

\textbf{10.} Explain why there does not exist $\lambda \in C$ such that $\lambda(2-3i,5+4i,-6+7i) = (12-5i,7+22i,32-9i)$\\ 

\begin{solution}
This is because $\lambda$ is a scalar, not a list/vector. It is one number that will be multiplied across every element of the vector. Because the vector on the right hand side is not a multiple of the vector on the left hand side, no such $\lambda$ exists to satisfy this. 
\end{solution}\\

\textbf{11.} Show that $(x+y)+z = x + (y+z)$ for all $x,y,z \in F^n$ \\ 

\begin{solution}
\begin{align*}
    & \text{LHS: } & \\
    & (x+y) + z = ((x_1, ... , x_n) +(y_1,...,y_n)) + (z_1,...z_n) & \\
    & (x_1 + y_1, ..., x_n + y_n) + (z_1,...z_n) & \\
    & (x_1 + y_1+z_1,...x_n+y_n+z_n) & \\
    & \text{RHS: } & \\
    & x + (y+z) = (x_1, ... , x_n) +((y_1,...,y_n) + (z_1,...z_n)) & \\
    & (x_1, ... , x_n) + (y_1 + z_1, ..., y_n + z_n) & \\
    & (y_1+ z_1 + x_1 ,...y_n+z_n+x_n) & \\
    & \text{By commutativity: } RHS = LHS
\end{align*}
\end{solution}

\textbf{12.} Show that (ab)x = a(bx) for all $x \in F^n$ and $a,b \in F$ \\

\begin{solution}
\begin{align*}
    & \text{LHS: } & \\
    & \text{Let } ab = c \rightarrow ab(x) = cx & \\
    & = (cx_1,...,cx_n) & \\
    & \text{RHS: } & \\
    & a(bx) = a(bx_1,...,b_xn) & \\
    & = (abx_1,...,abx_n) = (cx_1,...cx_n) &
\end{align*}
\end{solution}

\textbf{13.} Show that $1x = x$ for all $x \in F^n$ \\

\begin{solution}
\begin{align*}
    & 1x = 1(x_1,...,x_n) = (1x_1,...1x_n) = (x_1,...x_n) = x &
\end{align*}
\end{solution}

\textbf{14.} Show that $\lambda(x+y) = \lambda x + \lambda y$ for all $\lambda \in F$ and $x,y \in F^n$ \\

\begin{solution}
\begin{align*}
    &  LHS = \lambda(x+y) = \lambda(x_1+y_1,...x_n+y_n) & \\
    & = (\lambda(x_1 +y_1), ..., \lambda(x_n+y_n) = (\lambda x_1 + \lambda y_1,...,\lambda x_n + \lambda y_n) & \\
    & = \lambda(x_1,...,x_n) + \lambda(y_1,...,y_n) = \lambda(x) + \lambda(y)  = RHS &
\end{align*}
\end{solution}

\textbf{15.} Show that $(a+b)x = ax + bx$ for all $a,b \in F$ and $x \in F^n$ \\

\begin{solution}
\begin{align*}
    & \text{Let } a+b  = c \rightarrow LHS = cx & \\
    & = c(x_1,...x_n) = (cx_1,...,cx_n) = ((a+b)x_1,...,(a+b)(x_n)) & \\
    & (ax_1 + bx_1,...,ax_n + bx_n) = ax + bx = RHS
\end{align*}
\end{solution}


\subsection*{Exercises 1B: Definition of Vector Space}

\textbf{1. } Prove that $-(-\nu) = \nu$ for every $\nu \in V$ \\

\begin{solution}
\begin{align*}  \\
\text{Given } \nu - \mu = \nu +(-\mu)  \\
\nu - (-(-\nu)) = \nu + (-(-(-\nu))) \\
= \nu + (-\nu) = 0  \\
-\nu +(-1(-\nu)) = 0  \\
-(-\nu) = \nu
\end{align*}
\end{solution}

\textbf{2. } Suppose $a \in F, \nu \in V$ and $a\nu = 0$ Prove that $a = 0$ or $\nu = 0$ \\ 

\begin{solution}
\begin{align*}  \\
\text{Take the contrapositive, negate hypothesis and conclusion} \\
\text{By DeMorgan's law: } \neg(a=0 \lor \nu=0) = a\ne{0} \wedge \nu \ne 0  \\
\text{ where the conclusion becomes: } \neg(a\nu = 0) \implies a\nu\ne0\\
\text{The product of two non-zero elements of any } F^{n} \text{ cannot be 0} \\
\end{align*}
\end{solution}

\textbf{3. } 















%%%%%%%%%%%%%%%%%%%%% Bibliography %%%%%%%%%%%%%%%%%%%%%%%%%%%
\newpage
\bibliographystyle{plain}
\bibliography{citations/literature,citations/data}
\begin{enumerate}
    \item Axler, S. (2024). Linear Algebra Done Right (4th ed.). Springer.
\end{enumerate}





\end{document}  
